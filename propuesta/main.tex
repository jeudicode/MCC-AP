\documentclass[letterpaper,12pt]{article}
\usepackage[utf8]{inputenc}
\usepackage[spanish,es-tabla]{babel}
\decimalpoint
\usepackage{amsfonts}
\usepackage{float}
\usepackage{booktabs}

%\usepackage{heuristica}
%\usepackage[heuristica,vvarbb,bigdelims]{newtxmath}
\usepackage{graphicx}
\graphicspath{ {./} }

\usepackage{libertinus}
\usepackage[T1]{fontenc}

\usepackage[margin=1.3in]{geometry}
\usepackage{amsthm}
\usepackage{marvosym}
\usepackage{bm}

\renewcommand\qedsymbol{\Squarepipe}

\theoremstyle{definition}
\newtheorem{definition}{Definición}[section]
\newtheorem*{thm}{Teorema}


\setlength\parindent{0pt}

\newcounter{paragraphnumber}
\newcommand{\para}{%
  \vspace{10pt}\noindent{\bfseries\refstepcounter{paragraphnumber}\theparagraphnumber.\quad}%
}

%\setsecheadstyle{\large\bfseries}
%\setsubsecheadstyle{\bfseries}

\setlength\parindent{0pt}

\pagenumbering{gobble}


\usepackage{enumitem}
\setlist{nosep}

\usepackage{xcolor}

\usepackage{hyperref}
\hypersetup{
  colorlinks,
  linkcolor={red!50!black},
  citecolor={blue!50!black},
  urlcolor={green!50!black}
}

\usepackage{amssymb}
\usepackage{amsmath}

\begin{document}



\begin{center}
  {\large Aprendizaje Profundo}\\
  \vspace{0.2cm}
  {\large\bfseries Propuesta de artículo}\\
  \vspace{0.2cm}
  {\large PCIC - UNAM}\\
  \vspace{0.5cm}
  {\itshape 15 de octubre de 2020}\\
  \vspace{0.5cm}
  Diego de Jesús Isla López\\
  Saúl Iván Rivas Vega\\
  (\href{mailto:dislalopez@gmail.com}{\itshape dislalopez@gmail.com})\\
  (\href{mailto:diego.isla@comunidad.unam.mx}{\itshape diego.isla@comunidad.unam.mx})\\
\end{center}

Se plantea la reproducción de los experimentos realizados en el artículo \textit{Counterpoint by convolution} de Huang et al \cite{DBLP:journals/corr/abs-1903-07227}, en el cual presentan COCONET, un modelo entrenado para reconstruir composiciones musicales parciales; dicho de otra manera, se intenta crear una forma de generación musical más cercana a la manera en la que las personas realizan obras musicales; esto es, componiendo extractos independientes para después unirlos, en lugar de un proceso lineal. La arquitectura utilizada en el proyecto es la llamada NADE, propuesta en \cite{DBLP:journals/corr/UriaCGML16} por Uria et al. Como trabajo adicional se propone utilizar un esquema similar a COCONET para realizar reconstrucciones de imágenes, como lo presentado en \cite{DBLP:journals/corr/abs-1802-05751}. 


\bibliographystyle{plain}
\bibliography{referencias}
\end{document}

